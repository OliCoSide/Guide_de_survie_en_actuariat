% SECTION 1.4 TAUX INT�R�T NOMINAUX
Nous avons convenu que, par d�faut, le taux d'int�r�t est annuel.

Donc, un taux effectif de 8\%/an = a(t) = $(1,08)^t$
\n
Mais un taux nominal de 8\%/an = ???

\subsubsection{Exemple}
24\%/an capitalis� mensuellement
\n
$\frac{24\%}{12} = 2\% / mois \Rightarrow $ taux effectif mensuel de 2\%

\begin{gather*}
i^{(m)} \text{: taux d'int�r�t annuel (nominal), compos� $m$ fois par an} \\
\frac{i^{(m)}}{m} \text{: taux d'int�r�t effectif par $(\frac{1}{m})$ d'ann�e.} \\
a(t) = (1+i)^t = \left( 1+\frac{i^{(m)}}{m} \right)^{t \cdot m} \\
\Rightarrow a+i = \left( 1 + \frac{i^{(m)}}{m} \right)^m - 1 \\
\end{gather*}
Donc,
\begin{empheq} [box=\fcolorbox{black}{green}]{equation}
i = \left( 1 + \frac{i^{(m)}}{m} \right)^m - 1
\end{empheq}

\begin{empheq} [box=\fcolorbox{black}{green}]{equation}
i^{(m)} = m \left( (1 + i)^{\frac{1}{m}} - 1 \right)
\end{empheq}
o� $i^{(m)} < i$ et $ m>1$.

\subsubsection{Exemple 1.9 (modifi�)}
$i_{A}^{(2)}$ = 15,25\% et $i_{B}^{(12)} = 15\%$, $i_{A}^{(12)} = ?$
\begin{gather*}
\left( 1 + \frac{i_{A}^{(2)}}{2} \right)^2 = \left( 1 + \frac{i_{A}^{(12)}}{12} \right)^12 \\
\Rightarrow i_{A}^{(12)} = 12 \left[ \left(1 + \frac{i_{A}^{(2)}}{2} \right)^{2/12} - 1\right ] \\
i_{A}^{(12)} = 0,147869192 \\
i_{A}^{(12)} < i_{B}^{(12)}
\end{gather*}
% FIN SECTION 1.4 TAUX INT�R�T NOMINAUX ET EFFECTIFS