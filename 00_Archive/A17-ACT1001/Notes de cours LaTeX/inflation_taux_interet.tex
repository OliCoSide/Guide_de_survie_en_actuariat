% 1.6 Inflation et taux d'int�r�t	
IPC (\emph{CPI}) : Indice des prix � la consommation
\p
Contexte : tous les taux sont \textbf{effectifs annuels et compos�s} dans cette section
\p
$r$ : taux d'inflation $\Rightarrow$
\begin{tabular}{ll}
Ce qui co�te $X$ 	& � $t=0$ \\
Co�te $X(1+r)$		& � $t=1$ \\
\end{tabular}
\p
$i$ : taux d'int�r�t $\Rightarrow$
\begin{tabular}{ll}
Ce qui vaut $X$ 	& � $t=0$ \\
vaudra $X(1+i)$		& � $t=1$ \\
\end{tabular}
\p
Question � se poser : Quelle est l'augmentation r�elle du \textbf{pouvoir d'achat}?
\p
Que vaut $i_{r�el}$?
\p
En temps normal, on indiquerait que

\begin{gather*}
i = \frac{X(1+i)}{X} - 1
\end{gather*}

Sauf qu'on vient d'indiquer clairement qu'� $t=1$, $X$ est maintenant �gal � $X(1+r)$. Donc : 

\begin{gather*}
i_{r�el} = \frac{X(1+i)}{X(1+r)} - 1 = \frac{i-}{1+r} 
\end{gather*}

\subsubsection{Exemple}

La valeur d'une \emph{Porsche} de l'ann�e est fix�e � 125 000\$. Vous avez actuellement 115 000\$ en liquidit�, donc vous d�cidez d'investir dans un placement et d'attendre � l'an prochain pour acheter la voiture.
\p
a) quel est le taux d'int�r�t $i$ n�cessaire pour atteindre l'objectif, si l'IPC reste constant?

\begin{gather*}
115 000(1+\red{i}) = 125 000 \Rightarrow \red{i = 8,70\%}
\end{gather*}
\p
b) Si il y a une inflation ($r$) de 2\% sur le march� des voitures de luxe, quel devra �tre le taux d'int�r�t $i$.

\begin{gather*}
115000(1+i) = 125000(1+0,02) \Rightarrow i = \frac{125000(1,02)}{115000} - 1 \Rightarrow i = 10,87\%
\end{gather*}
\p
c) Quel est le taux d'int�r�t r�el ($i_{r�el}$) r�alis� par le placement s'il obtient l'objectif de b)?

\begin{gather*}
i_{r�el} = \frac{i-r}{1+r} = \frac{0,1087 - 0,02}{1,02} = 8,70\%
\end{gather*}

\subsubsection{Remarque}
On peut donc conclure que \hl{le taux d'int�r�t r�el est l'augmentation de notre pouvoir d'achat}, le taux d'int�r�t $i$ une fois l'inflation $r$ consid�r�e.

