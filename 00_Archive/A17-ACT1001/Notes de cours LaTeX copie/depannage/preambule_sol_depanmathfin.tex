% Préambule pour rédaction LaTeX

% Préparé par Gabriel Crépeault-Cauchon
% Si vous n'avez pas besoin de l'un des packages, simplement à indiquer un «%»
% au début
% -----------------------------------------------------------------------------

% Packages pour permettre les caractères accentués
\usepackage[utf8]{inputenc}
\usepackage[T1]{fontenc}
\usepackage{babel}
\usepackage{lmodern}

% Package pour aggrandir les marges d'un document
% \usepackage{fullpage}

% Packages mathématiques essentiels
\usepackage{amsmath,amsthm,amssymb,latexsym,amsfonts}
\usepackage{empheq}
\usepackage{numprint}


% Packages pour des graphiques avancés
\usepackage{graphicx}
\usepackage{pict2e}

% Package pour faire des listes plus avancées
\usepackage{enumitem}

% Packages pour créer des liens URL
\usepackage{hyperref}

% Package pour l'insertion de documents PDF à même un fichier LaTeX
\usepackage{pdfpages}

% Package pour éditer les couleurs du fichier
% \usepackage[svgnames]{xcolor}
\usepackage{color,soul}


% NOUVELLES COULEURS
\newcommand{\orange}{\textcolor{orange}}
\newcommand{\red}{\textcolor{red}}
\newcommand{\cyan}{\textcolor{cyan}}
\newcommand{\blue}{\textcolor{blue}}
\newcommand{\green}{\textcolor{green}}
\newcommand{\purple}{\textcolor{magenta}}
\newcommand{\yellow}{\textcolor{yellow}}


% Commandes pour certains symboles mathématiques...
\newcommand{\reels}{\mathbb{R}}
\newcommand{\entiers}{\mathbb{Z}}
\newcommand{\naturels}{\mathbb{N}}
\newcommand{\eval}{\biggr \rvert}
\usepackage{cancel}



% Package bclogo pour insérer des logos intéressant dans le fichier
\usepackage[tikz]{bclogo}
\usepackage{actuarialsymbol}
\usepackage{actuarialangle}

% Commandes utiles pour sauver du temps dans la rédaction

\newcommand{\n}{\newline}
\newcommand{\p}{\paragraph{}}
